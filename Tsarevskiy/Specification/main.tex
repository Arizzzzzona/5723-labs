\documentclass[12pt,a4paper]{article}
\usepackage{cmap}
\usepackage[utf8]{inputenc}
\usepackage[russian]{babel}
\begin {document}
\thispagestyle{empty}
\begin{center}
ГУАП

КАФЕДРА №52
\end{center}

\vspace{30mm}
ПРЕПОДАВАТЕЛЬ

\begin{tabular}{ |c|c|c| }
\hline
доц., к. т. н. & & Е.М. Линский \\\hline
должность, уч. ст., звание & дата, подпись & инициалы, фамилия \\\hline
\end{tabular}

\vspace{30mm}
\begin{center}
СПЕЦИФИКАЦИЯ

СЕРВЛЕТ ДЛЯ «СМИШНЫХ КАРТИНОК»
\end{center}

\vspace{10mm}
по курсу: ТЕХНОЛОГИИ ПРОГРАММИРОВАНИЯ

\vspace{50mm}
\hspace{8mm}РАБОТУ ВЫПОЛНИЛ
\begin{center}
\begin{tabular}{ |c|c|c| }
\hline
Студент гр. 5723 & & А.В. Царевский \\\hline
& дата, подпись & инициалы, фамилия \\
\hline
\end{tabular}
\vspace{15mm}

Санкт-Петербург 2019
\end{center}

\newpage
\begin{center}
\Large{\bf Сервлет для «СМИШНЫХ КАРТИНОК» }
\end{center}

Сервлет для "смишных картинок" представляет собой базу данных в
которой хранятся исходные изображения. Программа представлена в
cледующем виде: главная страница,на которой представленны последние добавленные картинки в виде ленты, слева список тредов(картинки по темам). Пользователю доступна возможность получить картинку.База данных с изображениями хранится на сервере. Роль сервера выполняеткомпьютер, на котором находится папка, в которой хранятся изображения. Изображения будут представленны в виде списка тем.

\newpage
\begin{center}
\Large{\bf Инструкция пользователя }
\end{center}
После открытия сайта пользователь будет на главнной странице,где можно скролить ленту с последними добаленными картинками или, нажимая на тему (тред), откроется страница с картинками.


\end{document}